\section{Related work}
\label{sec:related}




Deep learning approaches were initially described as black-box methods, meaning that not much were known about the reasoning and decisions of the created models. Much exertion have been applied to investigate the networks operation, whether by methodical experimentation~\cite{ilin17,kuo16,eigen14,zhang17} or visualization methods~\cite{simonyan13,zeiler14}. Those efforts provided more clarity of the deep models and characterized the learned features as complex concepts build from simpler ones. Also it demonstrates the learning progression from detailed to coarser representations as the scale and resolution reduce through the network.  When applied to object recognition task for instance, the raw pixel on the input layer is learned as segments and parts until the composition of an object concept on posterior layers, while the input scale reduces to a single feature vector on the output.  

The knowledge of the concept abstraction and learning progression allowed new research endeavors to explore  them in high-level tasks. For instance, three main architectures standout in recent years, namely: (i) Holistically-nested Edge Detection~(HED)~\cite{xie2015}; (ii) Convolutional Oriented Boundaries~(COB)~\cite{maninis2017}; and (iii) Rich Convolutional Features~(RCF)~\cite{liu2017}. These architectures explicitly explore the models of a traditional deep network to perform a certain high-level task, in which all extract side-outputs of the network and each present a different strategy to combine them . 

The HED network creates a side-output layer at each stage of the VGG16 network~\cite{simonyan2014} as boundary maps. In HED, each side-output is associated with a classifier in a deeply supervised scheme~\cite{lee2015} for the task of edge detection. This association inserts the fusion process in the network, attributing weights for each side-output that will be learned individually and determine its contribution on the final evaluation. The evaluation is performed by a cost-sensitive function to balance the bias towards non-edge pixels. The HED network significantly improved the performance in multiple datasets and the extended version~\cite{xie2017} also applies the network for the segmentation task. 

The authors in~\cite{cheng2016} use the edge maps created by the HED network alongside with other features such as brightness, color, gradient and variance to describe images. The goal of their proposal was to create an efficient framework for real-time segmentation, focused on a fusion strategy to update region features.


In the COB network, the authors also create edge maps from side activations, differing mainly from HED by the attribution to candidate contours orientation information and weights representing the contour strength. The contour orientations are estimated by approximation to known polygon segments and the segments weights are computed based on the candidate contour neighboring region used as a confidence measure. To combine the side outputs a non-linear function is used to regress both the segment weights and the orientation maps, creating region hierarchical trees by thresholding the contour strength. The network perform well in multiple tasks such as object proposal, object detection, semantic contour and segmentation.


Finally, the RCF network, not only creates multiple side-outputs, but also uses multiple scales of the images in the input layer. Differently from the HED network, RCF extract one side-output at each convolutional layer of VGG, arguing that  this could create more detailed representations and improve the network accuracy. The merging process is performed by a series of operations, comprising grouping by convolutions, element-wise sums, up-samplings, local loss functions and concatenations.  



%Regarding the fusion methods, in the aforementioned related strategies, the HED fusion process is also inserted in the network, in which it is attributed weights for each side output that will be learned individually and determine its contribution on the final evaluation. The RCF instead, adopted a $1\times1$ convolution after each output, followed by a element-wise sum, an up-sampling and a loss function. The values are then concatenated and pass through a $1\times1$ convolution for the final evaluation. Finally, the COB network evaluates the The side outputs and the contour orientation maps are evaluated individually creating multiple proposal that could be thresholded based on the confidence value at later stage.  



%\begin{figure*}[!ht]
%\begin{center}
%\begin{tabular}{l}
%(1a) Side outputs extracted at each stage\\
%\resizebox{\textwidth}{!}{%
\begin{tikzpicture}[
std/.style={
  draw,
  text width=1.5cm,
  align=center,
  font=\strut\sffamily
  },
rnd/.style={
  draw=#1,
  rounded corners=8pt,
  line width=1pt,
  align=center,
  text width=2cm,
  minimum height=1cm,
  font=\strut\sffamily
  },
vac/.style={
  text width=2.5cm,
  align=center,
  font=\strut\sffamily
  },
ar/.style={
  ->,
  >=latex
  },
node distance=0.5cm and .5cm    
]



%The nodes for the left
\node[rnd] (inp)
  {Input layer};
\node[rnd=gruen_4b,right=of inp] (conv1)
  {Convolution};  
\node[rnd=gruen_4b,right=of conv1] (conv2)
  {Convolution};  
\node[draw,dashed,gruen_4b,inner sep=8pt,fit={(conv1) (conv2)}]
  (fit) {};
\node[rnd,right=1cm of conv2] (pool1)
  {Pooling};
\node[vac,left=-1cm of pool1, yshift=-1.5cm] (h1)
  {$\mathcal{H}_1$}; 
\node[vac,gruen_4b,left=-1cm of conv2, yshift=1cm] (s1)
  {Stage 1};   
\node[vac,right=-1cm of pool1] (p)
  {$\mathbf{\cdots}$}; 
  
\node[rnd=gruen_4b,right=-.7cm of p] (conv3)
  {Convolution};  
\node[rnd=gruen_4b,right=of conv3] (conv4)
  {Convolution};  
\node[rnd=gruen_4b,right=of conv4] (conv5)
  {Convolution};    
\node[draw,dashed,gruen_4b,inner sep=8pt,fit={(conv3) (conv5)}]
  (fit2) {};
\node[rnd,right=1cm of conv5] (pool2)
  {Pooling};
\node[vac,left=-1cm of pool2, yshift=-1.5cm] (h2)
  {$\mathcal{H}_n$};
  \node[vac,gruen_4b,left=-3cm of conv4, yshift=1cm] (s2)
  {Stage $N$};   
\node[vac,right=-1cm of pool2] (p2)
  {$\mathbf{\cdots}$};   

%Aux  
\coordinate (aux1) at ( $ (conv2.east|-conv2.west)!0.6!(pool1.west) $ );
\coordinate (aux2) at ( $ (conv5.east|-conv5.west)!0.6!(pool2.west) $ );

\draw[ar]  (inp) -- (conv1); 
\draw[ar]  (conv1) -- (conv2); 
\draw[ar]  (conv2) -- (pool1); 
\draw[ar,left=-1cm of pool1, yshift=-1.5cm]  (aux1) -- (h1); 
\draw[ar]  (conv3) -- (conv4);
\draw[ar]  (conv4) -- (conv5); 
\draw[ar]  (conv5) -- (pool2); 
\draw[ar,left=-1cm of pool2, yshift=-1.5cm]  (aux2) -- (h2); 
\end{tikzpicture}%
}\\
%(1b) Side outputs extracted at each convolutional layer\\
%\resizebox{\textwidth}{!}{%
\begin{tikzpicture}[
std/.style={
  draw,
  text width=1.5cm,
  align=center,
  font=\strut\sffamily
  },
rnd/.style={
  draw=#1,
  rounded corners=8pt,
  line width=1pt,
  align=center,
  text width=2cm,
  minimum height=1cm,
  font=\strut\sffamily
  },
vac/.style={
  text width=2.5cm,
  align=center,
  font=\strut\sffamily
  },
ar/.style={
  ->,
  >=latex
  },
node distance=0.5cm and .5cm    
]



%The nodes for the left
\node[rnd] (inp)
  {Input layer};
\node[rnd=lila_10b,right=of inp] (conv1)
  {Convolution};  
\node[rnd=lila_10b,right=of conv1] (conv2)
  {Convolution};  
\node[rnd,right=1cm of conv2] (pool1)
  {Pooling};
\node[vac,right=-1cm of pool1] (p)
  {$\mathbf{\cdots}$}; 
  
\node[rnd=lila_10b,right=-.7cm of p] (conv3)
  {Convolution};  
\node[rnd=lila_10b,right=of conv3] (conv4)
  {Convolution};  
\node[rnd=lila_10b,right=of conv4] (conv5)
  {Convolution};    
\node[rnd,right=1cm of conv5] (pool2)
  {Pooling};
\node[vac,right=-1cm of pool2] (p2)
  {$\mathbf{\cdots}$};   

\node[vac,left=-1cm of conv2, yshift=-1.5cm] (h1)
  {$\mathcal{H}_1$};
\node[vac,left=-1cm of pool1, yshift=-1.5cm] (h2)
  {$\mathcal{H}_2$};    
\node[vac,left=-1cm of conv4, yshift=-1.5cm] (h3)
  {$\mathcal{H}_{m-2}$};
\node[vac,left=-1cm of conv5, yshift=-1.5cm] (h4)
  {$\mathcal{H}_{m-1}$};
\node[vac,left=-1cm of pool2, yshift=-1.5cm] (h5)
  {$\mathcal{H}_m$};


%Aux  
\coordinate (aux1) at ( $ (conv1.east|-conv1.west)!0.3!(conv2.west) $ );
\coordinate (aux2) at ( $ (conv2.east|-conv2.west)!0.6!(pool1.west) $ );
\coordinate (aux3) at ( $ (conv3.east|-conv3.west)!0.3!(conv4.west) $ );
\coordinate (aux4) at ( $ (conv4.east|-conv4.west)!0.3!(conv5.west) $ );
\coordinate (aux5) at ( $ (conv5.east|-conv5.west)!0.6!(pool2.west) $ );

\draw[ar]  (inp) -- (conv1); 
\draw[ar]  (conv1) -- (conv2); 
\draw[ar]  (conv2) -- (pool1); 
\draw[ar]  (conv3) -- (conv4);
\draw[ar]  (conv4) -- (conv5); 
\draw[ar]  (conv5) -- (pool2); 

\draw[ar,left=-1cm of pool1, yshift=-1.5cm]  (aux1) -- (h1);
\draw[ar,left=-1cm of pool2, yshift=-1.5cm]  (aux2) -- (h2); 
\draw[ar,left=-1cm of pool2, yshift=-1.5cm]  (aux3) -- (h3);
\draw[ar,left=-1cm of pool2, yshift=-1.5cm]  (aux4) -- (h4);
\draw[ar,left=-1cm of pool2, yshift=-1.5cm]  (aux5) -- (h5);   

\end{tikzpicture}%
}\\
%\end{tabular}%
%\caption{Illustration for  the proposed side outputs extraction one following the HED model~(a) at each stage of the VGG network and the other following RCF model~(b) at each convolutional layer}
%\end{center}
%\label{fig:methods}
%\end{figure*}
