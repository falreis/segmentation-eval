\section{Introduction}
\label{sec:intro}

Image segmentation refers to the partition of an image into a set of regions representing  meaningful areas, and it is considered as a challenging semantic task, aiming to determine and group uniform regions for analysis. According to~\cite{DOMINGUEZ}, to create an adequate segmented image it is necessary that the output presents some fundamental characteristics, such as: (i) region uniformity and homogeneity in  its features, such as gray level, color or texture; (ii) region continuity, without holes; (iii) significant difference to adjacency regions; and (iv) spatial accuracy with smooth boundaries and without raggedness. Image segmentation is still an active topic of research and, usually, it could be divided in two stages~\cite{guigues06}: (i) low-level analysis, which evaluate the pixel characteristics, neighboring relation and it is ideally uncommitted in terms of position, orientation, size and contrast; and (ii) high-level analysis, which maps the low-level characteristics to fulfill the task.  

Recently, deep learning approaches have drastically changed the computational paradigm for visual tasks. The main advantage of deep learning algorithms is that it does not require an engineered model to operate, meaning that they are capable of learning not only the features to represent the data but also the models to describe it~\cite{goodfellow16}. Facing with this new paradigm, researches initially replaced  hand-craft features in the low-level analysis by the features learned in deep models~\cite{farabet2013,simonyan2014,lee2015}, which mostly achieved the desirable results. More recently, there are many proposals which explore the learned model for the high-level analysis in order to create maps from the outputs of different layers in a deep learning network~\cite{xie2017,cheng2016,maninis2017,liu2017}. 

%% IT IS IMPORTANT TO DEFINE CONCEPT...
One challenge on the latter strategy is how to combine the output from distinct layers, considering that they are presented with different sizes and could represent different concepts. In this work, we propose some strategies to combine the outputs from different layers by using simple merging functions in order to explore useful behavior in the learning process. We also study of the amount of combined side outputs which are necessary to create a viable region proposition for the task of road image segmentation. Moreover, we propose the use of a post-processing filtering based on mathematical morphology idempotent functions~\cite{najman13} in order to remove some underisable small segments.
%, more specifically,  an area opening mathematical morphology idempotent functions~\cite{najman13} to better cope with the fundamental characteristics of an ideal segmented image.  

The remainder of this work is organized as follows. In Section~\ref{sec:related}, some related works are described in order to characterize the hierarchy of concepts in deep models. In Section~\ref{sec:method}, the proposed method as well the reasoning and strategies are presented. In Section~\ref{sec:experiments}, a quantitative and qualitative assessment are done. And, finally, in Section~\ref{sec:conclusion}, some conclusions are drawn.


