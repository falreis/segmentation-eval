\documentclass[conference]{IEEEtran}
\IEEEoverridecommandlockouts
% The preceding line is only needed to identify funding in the first footnote. If that is unneeded, please comment it out.
\usepackage{cite}
\usepackage{amsmath,amssymb,amsfonts}
\usepackage{algorithmic}
\usepackage{graphicx}
\usepackage{textcomp}
\usepackage{xcolor}
\usepackage{hyperref} %url
\usepackage{comment}
%\usepackage[caption=false]{subfig}
\usepackage{booktabs}
\usepackage{tikz}
\usepackage{pgfplots}
\usepackage{pgfplotstable}
\usepackage{filecontents}
\usetikzlibrary{snakes,arrows,shapes,backgrounds, positioning,fit,calc}
\usetikzlibrary{decorations.text}
\usetikzlibrary{decorations.pathmorphing}





\input{config/colors}


\ifCLASSOPTIONcompsoc
\usepackage[caption=false,font=normalsize,labelfont=sf,textfont=sf]{subfig}
\else
\usepackage[caption=false,font=footnotesize]{subfig}
\fi

\usetikzlibrary{external}
\tikzexternalize[prefix=tikz/]
\tikzexternalize[prefix=tikz/,shell escape=-enable-write18]
\tikzset{external/system call= {pdflatex %-save-size=80000 
%                           -pool-size=10000000 
 %                          -extra-mem-top=50000000 
 %                          -extra-mem-bot=10000000 
  %                         -main-memory=90000000 
                           \tikzexternalcheckshellescape 
                           -halt-on-error 
                           -interaction=nonstopmode %-shell-escape
                           -jobname "\image" "\texsource"}} 
                           
%\usepackage{todonotes}
%\newcommand{\remSilvio}[1]{\todo[color=green]{{\tiny [SG]} {\scriptsize #1\par}}}
%\newcommand{\remSilvioi}[1]{\todo[color=green,inline]{{\tiny [SG]} {\scriptsize #1\par}}}

%\newcommand{\remSimon}[1]{\todo[color=blue]{{\tiny [SM]} {\scriptsize #1\par}}}
%\newcommand{\remSimoni}[1]{\todo[color=blue,inline]{{\tiny [SM]} {\scriptsize #1\par}}}

%\newcommand{\remEwa}[1]{\todo[color=yellow]{{\tiny [E]} {\scriptsize #1\par}}}
%\newcommand{\remEwai}[1]{\todo[color=yellow,inline]{{\tiny [E]} {\scriptsize #1\par}}}                        
\newcommand{\remEwai}[1]{{\color{violet}{\tiny [E]} {\scriptsize #1}}}    
                              
\begin{document}

%\title{Combining Side Outputs of Convolutional Neural Network for Road Segmentation
\title{Combining convolutional side-outputs for road image segmentation
\thanks{The authors are grateful to FAPEMIG (PPM 00006-16), CNPq (Universal 421521/2016-3 and PQ 307062/2016-3), CAPES (MAXIMUM STIC-AmSUD 048/14) and PUC Minas for the financial support to this work.}
}

\begin{comment}
\author{
\IEEEauthorblockN{Felipe Augusto Lima Reis}
\IEEEauthorblockA{\textit{Audio-Visual Information Processing Laboratory} \\
\textit{Pontifical Catholic University of Minas Gerais}\\
Belo Horizonte, Minas Gerais, Brazil\\
falreis@sga.pucminas.br}
\and
\IEEEauthorblockN{Raquel Almeida}
\IEEEauthorblockA{\textit{Audio-Visual Information Processing Laboratory} \\
\textit{Pontifical Catholic University of Minas Gerais}\\
Belo Horizonte, Minas Gerais, Brazil \\
raquel.almeida.685026@sga.pucminas.br}
\and
\IEEEauthorblockN{Silvio Jamil F. Guimar\~aes}
\IEEEauthorblockA{\textit{Audio-Visual Information Processing Laboratory} \\
\textit{Pontifical Catholic University of Minas Gerais}\\
Belo Horizonte, Minas Gerais, Brazil \\
sjamil@pucminas.br}
\and
\IEEEauthorblockN{Simon Malinowski}
\IEEEauthorblockA{\textit{Linkmedia}-- Univ Rennes, Inria, CNRS, IRISA \\
Rennes, France \\
simon.malinowski@irisa.fr}
\and
\IEEEauthorblockN{Zenilton K. G. do Patroc\'inio Jrs}
\IEEEauthorblockA{\textit{Audio-Visual Information Processing Laboratory} \\
\textit{Pontifical Catholic University of Minas Gerais}\\
Belo Horizonte, Minas Gerais, Brazil \\
zenilton@pucminas.br}
}
\end{comment}

\author{
\IEEEauthorblockN{
Felipe A. L. Reis\textsuperscript{\dag}, Raquel Almeida\textsuperscript{\dag}, Simon Malinowski\textsuperscript{*}, Ewa Kijak\textsuperscript{*} \\ Silvio Jamil F. Guimar\~aes\textsuperscript{\dag} and Zenilton K. G. do Patroc\'inio Jr.\textsuperscript{\dag}
}
\\
\IEEEauthorblockA{
\textsuperscript{\dag}\textit{Audio-Visual Information Processing Laboratory} -- \textit{Pontifical Catholic University of Minas Gerais} \\ 
Belo Horizonte, Minas Gerais, Brazil \\
\{falreis, raquel.almeida.685026\}@sga.pucminas.br, \{sjamil, zenilton\}@pucminas.br
\\ \\
\textsuperscript{*}\textit{Linkmedia} -- Univ Rennes, Inria, CNRS, IRISA\\
Rennes, France \\
\{simon.malinowski, ewa.kijak\}@irisa.fr
}
}

\maketitle

\begin{abstract}
\color{white}Image segmentation refers to the partition of an image into a set of regions representing meaningful areas and it is an active topic of research.  In this work it is proposed to explore the learned model in a deep architecture, presenting a strategy to combine side outputs extracted at different layers of the network. It is also proposed to study the impact of the amount of side outputs extracted and the impact of a post-processing strategy using mathematical morphology. Experiments demonstrated that proposed approach is viable and  achieve results comparable or superior to the state-of-the-art for the public available KITTI Road/Lane Dataset for image segmentation.
\end{abstract}

\begin{IEEEkeywords}
convolutional neural network, image segmentation, mathematical morphology, CNN, region detection
\end{IEEEkeywords}

\section{Introduction}
\label{sec:intro}

Image segmentation refers to the partition of an image into a set of regions representing  meaningful areas. It is considered a challenging semantic task, aiming to determine and group uniform regions for analysis. According to~\cite{DOMINGUEZ}, to create an adequate segmented image it is necessary that the output presents some fundamental characteristics, such as: (i) region uniformity and homogeneity in  its features, such as gray level, color or texture; (ii) region continuity, without holes; (iii) significant difference to adjacency regions; and (iv) spacial accuracy with smooth boundaries and without raggedness. 

Image segmentation is an active topic of research and in a typical approach the procedure could be divided in two stages~\cite{guigues06}: (i) low-level analysis, which evaluate the pixel characteristics, neighboring relation and it is ideally uncommitted in terms of position, orientation,
size and contrast; and (ii) high-level analysis, which maps the low-level characteristics to fulfill the task.  

Recently, the deep learning approach drastically changed the computational paradigm for visual tasks. The main advantage of deep learning algorithms is that it does not require an engineered model to operate, meaning that they are capable of learning not only the features to represent the data but also the models to describe it~\cite{goodfellow16}. Facing this new paradigm, researches initially replaced  hand-engineered features in the low-level analysis by the features learned in deep models~\cite{farabet2013,simonyan2014,lee2015}, which mostly achieve the desirable characteristics. More recently, there is many proposals exploring the learned model for the high-level analysis, creating maps from the outputs of different layers in a deep learning network~\cite{xie2017,cheng2016,maninis2017,liu2017}. 

One challenge on the latter strategy is how to combine the output from distinct layers, considering that they are presented with different sizes and could represent different concepts. In this work, it is presented strategies to combine the outputs from different layers, by using simple merging functions that explore useful behavior in the learning process. It is also studied the amount of combined outputs necessary to create a viable region proposition for the task of image segmentation. In addition, it is also presented a post-processing filtering step using mathematical morphology idempotent functions~\cite{najman13} to better cope with the fundamental characteristics of an ideal segmented image.  

The remainder of this work is organized as it follows:  In Section~\ref{sec:related} it is characterized the hierarchy of concepts in deep models related work exploring theses models for high-level tasks. In Section~\ref{sec:method} it is presented the reasoning and strategies proposed in this work. In Section~\ref{sec:experiments} it is presented the dataset description, the experimental setup and results for the experiments. And, finally, in Section~\ref{sec:conclusion} the conclusions are laid out.


%\section{Hierarchies in deep models}
\label{sec:deep-hier}

Deep learning approaches were initially described as black-box methods, meaning that not much were known about the reasoning and decisions of the created models. Much exertion have been applied to investigate the networks operation, whether by methodical experimentation~\cite{ilin17,kuo16,eigen14,zhang17} or visualization methods~\cite{simonyan13,zeiler14}. Those efforts provided more clarity of the hierarchical aspects of the deep features, which allowed researches to explore these aspects in their endeavors. The  hierarchies learned in deep models are categorized as complex concepts build from simpler ones. When applied for object recognition task for instance, the raw pixel on the input layer is learned as segments and parts until the composition of an object concept at the final layer.  

These hierarchies could be particularly observed in convolutional networks, which are a stacked composition of three main layers, namely: (i) convolution; (ii) pooling; and (iii) non-linear activation. In \cite{ilin17} the authors directly assessed the hierarchy of concepts in convolutional networks, analyzing the knowledge representation and the network abstraction at each type of layer.  The authors are capable to demonstrate the generic aspect at earlier stages and the specialization at later layers. The findings are conformed with the expected behavior of convolutional networks, but it is possible to observe that most of the learned abstraction is due the convolutional layers and that the pooling and non-linear layers rarely contribute for increasing  the abstraction level.


\section{Related work}
\label{sec:related}


In the earlier years of the deep learning resurgence, the proposal in~\cite{farabet2013} tackles the task of scene parsing---segmentation task applied for each pixel of the image, aiming to group pixels composing all the identifiable objects in the scene---using hierarchical trees and deep features alongside. Images are used as input for a convolutional network to extract deep features from multiple scales of the images, and in parallel to construct a segmentation tree representing in its nodes dissimilarities of neighboring pixels. The tree nodes are used to pool the correspondent deep features to be processed by a classifier. The classifier scores are used to create histogram of object classes for each node of the segmentation tree, and the final parsing proposal is built using the class entropy distribution for selecting the nodes that cover the entire image.   This proposal uses hierarchical trees as auxiliary and external structures for the deep model. 

In exploring the hierarchies within the deep model, three main architectures standout in recent years, namely: (i) Holistically-nested Edge Detection~(HED)~\cite{xie2015}; (ii) Convolutional Oriented Boundaries~(COB)~\cite{maninis2017}; and (iii) Rich Convolutional Features~(RCF)~\cite{liu2017}. Those networks explicit explore the hierarchies by extracting side outputs of traditional convolutional networks to create boundary maps. The process to fuse these maps is also inserted in the network, in which it is attributed weights for each map that will be learned individually and determine its contribution on the final evaluation. 

To the best of our knowledge, the first network exploring this strategy was HED~(extended version in~\cite{xie2017}), which applied the boundary maps for the boundary detection task, aiming to identify the limits separating uniform regions. The HED network create an side-output layer at each stage of the VGG16 network~\cite{simonyan2014}, in which the stages are composed by two Convolution+ReLU layers followed by a Max Pooling layer. In HED, each side-output layer is associated with a classifier in a deeply supervised scheme~\cite{lee2015}. The layers create edge maps, which are scaled and fused at the end, to be evaluate by a cost-sensitive function to balance the bias towards not-boundary pixels. The HED network significantly improved the performance in multiple datasets. The extended version also applied the network for the segmentation task. The authors in~\cite{cheng2016} use the edge maps created by the HED network alongside with other features such as brightness, colors, gradient and variance to describe images. The goal of their proposal was to create an efficient framework to be used as real-time segmentation system, focused on a fusion strategy to update region features.


In the COB network, the authors also create edge maps from side activations, differing maily from HED by the attribution to candidate contours the orientation information and weights representing the contour strength. The contour orientations are estimated by approximation to known polygon segments and are used to create segmentations hierarchies. The segments weights are computed based on the candidate contour neighboring region to measure the confidence that the candidate is a boundary line. The weights are thresholded to determine the granularity of the segment when creating the segmentation hierarchy. The network perform well in multiple tasks such object proposal, object detection, semantic contour and segmentation.


Finally, the RCF network applied in the boundary detection task, which differ from HED by three main modifications. The first regards the input layer, in which it is used pyramids to create multiple scales of the images. The scaled images are later interpolated in the output layer, similar to~\cite{farabet2012}. The second modification regards the number of side output maps. RCF creates a side output at each Convolutial+ReLU layer of the VGG16 network, which is believed to create more detailed representations and improve the network accuracy. The last modification is in the loss function and the ground-truth of the datasets. In the ground-truth images, pixels are weighted based on a vote among multiple human-annotated values. Any pixel that not achieve a confidence vote value is disregarded by the loss function in the network. The goal is to reduce inconsistencies in the fallible human annotations and mitigate the network confusion in controversial pixels.  


\begin{figure*}[!ht]
\begin{center}
\begin{tabular}{l}
(1a) Side outputs extracted at each stage\\
\resizebox{\textwidth}{!}{%
\begin{tikzpicture}[
std/.style={
  draw,
  text width=1.5cm,
  align=center,
  font=\strut\sffamily
  },
rnd/.style={
  draw=#1,
  rounded corners=8pt,
  line width=1pt,
  align=center,
  text width=2cm,
  minimum height=1cm,
  font=\strut\sffamily
  },
vac/.style={
  text width=2.5cm,
  align=center,
  font=\strut\sffamily
  },
ar/.style={
  ->,
  >=latex
  },
node distance=0.5cm and .5cm    
]



%The nodes for the left
\node[rnd] (inp)
  {Input layer};
\node[rnd=gruen_4b,right=of inp] (conv1)
  {Convolution};  
\node[rnd=gruen_4b,right=of conv1] (conv2)
  {Convolution};  
\node[draw,dashed,gruen_4b,inner sep=8pt,fit={(conv1) (conv2)}]
  (fit) {};
\node[rnd,right=1cm of conv2] (pool1)
  {Pooling};
\node[vac,left=-1cm of pool1, yshift=-1.5cm] (h1)
  {$\mathcal{H}_1$}; 
\node[vac,gruen_4b,left=-1cm of conv2, yshift=1cm] (s1)
  {Stage 1};   
\node[vac,right=-1cm of pool1] (p)
  {$\mathbf{\cdots}$}; 
  
\node[rnd=gruen_4b,right=-.7cm of p] (conv3)
  {Convolution};  
\node[rnd=gruen_4b,right=of conv3] (conv4)
  {Convolution};  
\node[rnd=gruen_4b,right=of conv4] (conv5)
  {Convolution};    
\node[draw,dashed,gruen_4b,inner sep=8pt,fit={(conv3) (conv5)}]
  (fit2) {};
\node[rnd,right=1cm of conv5] (pool2)
  {Pooling};
\node[vac,left=-1cm of pool2, yshift=-1.5cm] (h2)
  {$\mathcal{H}_n$};
  \node[vac,gruen_4b,left=-3cm of conv4, yshift=1cm] (s2)
  {Stage $N$};   
\node[vac,right=-1cm of pool2] (p2)
  {$\mathbf{\cdots}$};   

%Aux  
\coordinate (aux1) at ( $ (conv2.east|-conv2.west)!0.6!(pool1.west) $ );
\coordinate (aux2) at ( $ (conv5.east|-conv5.west)!0.6!(pool2.west) $ );

\draw[ar]  (inp) -- (conv1); 
\draw[ar]  (conv1) -- (conv2); 
\draw[ar]  (conv2) -- (pool1); 
\draw[ar,left=-1cm of pool1, yshift=-1.5cm]  (aux1) -- (h1); 
\draw[ar]  (conv3) -- (conv4);
\draw[ar]  (conv4) -- (conv5); 
\draw[ar]  (conv5) -- (pool2); 
\draw[ar,left=-1cm of pool2, yshift=-1.5cm]  (aux2) -- (h2); 
\end{tikzpicture}%
}\\
(1b) Side outputs extracted at each convolutional layer\\
\resizebox{\textwidth}{!}{%
\begin{tikzpicture}[
std/.style={
  draw,
  text width=1.5cm,
  align=center,
  font=\strut\sffamily
  },
rnd/.style={
  draw=#1,
  rounded corners=8pt,
  line width=1pt,
  align=center,
  text width=2cm,
  minimum height=1cm,
  font=\strut\sffamily
  },
vac/.style={
  text width=2.5cm,
  align=center,
  font=\strut\sffamily
  },
ar/.style={
  ->,
  >=latex
  },
node distance=0.5cm and .5cm    
]



%The nodes for the left
\node[rnd] (inp)
  {Input layer};
\node[rnd=lila_10b,right=of inp] (conv1)
  {Convolution};  
\node[rnd=lila_10b,right=of conv1] (conv2)
  {Convolution};  
\node[rnd,right=1cm of conv2] (pool1)
  {Pooling};
\node[vac,right=-1cm of pool1] (p)
  {$\mathbf{\cdots}$}; 
  
\node[rnd=lila_10b,right=-.7cm of p] (conv3)
  {Convolution};  
\node[rnd=lila_10b,right=of conv3] (conv4)
  {Convolution};  
\node[rnd=lila_10b,right=of conv4] (conv5)
  {Convolution};    
\node[rnd,right=1cm of conv5] (pool2)
  {Pooling};
\node[vac,right=-1cm of pool2] (p2)
  {$\mathbf{\cdots}$};   

\node[vac,left=-1cm of conv2, yshift=-1.5cm] (h1)
  {$\mathcal{H}_1$};
\node[vac,left=-1cm of pool1, yshift=-1.5cm] (h2)
  {$\mathcal{H}_2$};    
\node[vac,left=-1cm of conv4, yshift=-1.5cm] (h3)
  {$\mathcal{H}_{m-2}$};
\node[vac,left=-1cm of conv5, yshift=-1.5cm] (h4)
  {$\mathcal{H}_{m-1}$};
\node[vac,left=-1cm of pool2, yshift=-1.5cm] (h5)
  {$\mathcal{H}_m$};


%Aux  
\coordinate (aux1) at ( $ (conv1.east|-conv1.west)!0.3!(conv2.west) $ );
\coordinate (aux2) at ( $ (conv2.east|-conv2.west)!0.6!(pool1.west) $ );
\coordinate (aux3) at ( $ (conv3.east|-conv3.west)!0.3!(conv4.west) $ );
\coordinate (aux4) at ( $ (conv4.east|-conv4.west)!0.3!(conv5.west) $ );
\coordinate (aux5) at ( $ (conv5.east|-conv5.west)!0.6!(pool2.west) $ );

\draw[ar]  (inp) -- (conv1); 
\draw[ar]  (conv1) -- (conv2); 
\draw[ar]  (conv2) -- (pool1); 
\draw[ar]  (conv3) -- (conv4);
\draw[ar]  (conv4) -- (conv5); 
\draw[ar]  (conv5) -- (pool2); 

\draw[ar,left=-1cm of pool1, yshift=-1.5cm]  (aux1) -- (h1);
\draw[ar,left=-1cm of pool2, yshift=-1.5cm]  (aux2) -- (h2); 
\draw[ar,left=-1cm of pool2, yshift=-1.5cm]  (aux3) -- (h3);
\draw[ar,left=-1cm of pool2, yshift=-1.5cm]  (aux4) -- (h4);
\draw[ar,left=-1cm of pool2, yshift=-1.5cm]  (aux5) -- (h5);   

\end{tikzpicture}%
}\\
\end{tabular}%
\caption{Illustration for  the proposed side outputs extraction one following the HED model~(a) at each stage of the VGG network and the other following RCF model~(b) at each convolutional layer}
\end{center}
\label{fig:methods}
\end{figure*}
\section{Hierarchical maps in convolutional neural networks}

This work present strategies to merge hierarchical maps created from  outputs of different layers of a convolutional network.{\color{green}verificar se eh plagio}. In a convolutional network each layer is a three-dimensional array of size $h \times w \times d$, where $h$ and $w$ are spatial dimensions  and $d$ is the feature, channel or stride dimension. The first layer is the input image, with pixel size $h \times w$ and $d$ color channels. Locations in higher layers correspond to the locations in the image they are path-connected to, which are called their receptive fields. Convolutional networks are built on translation invariance and their basic components (convolution, pooling, and activation functions) operate on local input regions and depend only on relative spatial coordinates. 

The convolutional network model used in this work is the VGG network~\cite{simonyan2014},proposed in 2014 as one of the first attempts to create deeper models for the task of object recognition. The architecture is a composition of multiple stacked convolutional layers, in which the receptive fields and stride have a fixed $3\times3\times1$ size. Following each two or three layers of convolution is placed a max-pooling layer. Also, all hidden layers are supplied with a ReLU non-linear rectification.

As demonstrated in bla bla layers bla bla hierarchies bla bla for binary segmentation bla bla bla

Formally, let $\mathit{S}=\{(\mathit{X_n,Y_n}), \mathit{n}=1,...,\mathit{N}\}$ be the training input set for the network, in which $\mathit{X_n}$ is a set of $\mathit{N}$ images with three color channels and $\mathit{Y_n}$ the set of $\mathit{N}$ labels associated with each image with values belonging to $\{0,1\}$. Consider also $\mathbf{W}$ the layer set of parameters in which
$\mathbf{w}=\{\mathbf{w}_1,...,\mathbf{w}_M\}$ is the associated weights for each one of the $\mathit{M}$ side output maps. The objective function for training the weights for the $\ell_{side}$ image map could be defined as:
\begin{equation}
\mathcal{L}(\mathbf{W},\mathbf{w})=\sum_{m=1}^M\alpha_m\ell_{side}^{(m)}(\mathbf{W},\mathbf{w}_m)
\end{equation}
\section{Experiments}
\label{sec:experiments}

Experiments were conducted in the KITTI Road/Lane dataset, part of KITTI Vision Benchmarking Suite~\cite{KITTI}. The dataset  contains  images for road and lane estimation for the task of image segmentation. It consists of 289 training and 290 test RGB images, with the size of 1242 pixels width and 375 pixels height. \remEwai{\textbf{What is the resolution of the images?}} The ground-truth is manually annotated for two different road types: (i) road, road area composing all lanes; and (ii) lane, lane the vehicle is currently driving on. Some images contains also sidewalks, that was not evaluated in this paper - sidewalks were classified as background. It is important to notice that the ground-truth is only available for training set and the test evaluation should be performed using KITTI Server.

In this work, only the road ground-truths is used and the lane annotations are ignored. This dataset contains the same image with different ground-truths for lane and road estimation. \remEwai{Should be good to show examples of images and groundtruths from the dataset} \remFeli{I think we won't have enought space in this paper}  Then, we prefer to use the road estimation and build the classifier on a binary problem~(road and background). The road type is divided in three different categories of road scenes, namely: (i) uu\_road, urban unmarked; (ii) um\_road, urban marked; and (ii) umm\_road, urban multiple marked lanes.  

To increase the number of images in the training set, a data augmentation procedure is performed. The following transformations were applied: pepper/salt noise, horizontal flipping (mirror), contrast change, brightness change, noise shadow and random rain/snow. Procedures that would create undesired behavior, such as the road in the sky and distortions that would change the nature of the objects in the scene, such as cars and pedestrians were avoided. Augmentation procedures resulted in 2601 images, divided in 2080 samples for training and 521 samples for validation (about 20\%). 


\subsection{Experimental setup}
   
Our networks were build using using Keras \cite{chollet2015keras} with Tensorflow \cite{tensorflow2015-whitepaper}. We used a pre-trained VGG16 model to initialize the weights. Also, we use SGD optimization with learning rate set to 1e-3, decay of 5e-6 and momentum of 0.95. The default batch size contains 16 images. All training experiments were performed in GeForce GTX 1080 8GB GPU.

%For simplicity, in the remaining of this work, the network using the side outputs extracted at each stage of the VGG will be called Stage Layer Outputs~(\textbf{SLO}) and it is composed by $n=5$ side outputs. Similarly, for the side outputs extracted at each convolutional layer, it will be called All Layers Outputs~(\textbf{ALO}) and it is composed by $n=13$ side outputs. For comparison, it is also defined a network similar to VGG, with only the final output, without any side outputs, called No Side Outputs~(\textbf{NSO}).
The \textbf{SLO} network is composed of $n=5$ side outputs, and the \textbf{ALO} network is composed of $n=13$ side outputs. 
The operations to combine side outputs are presented in the name of the methods. The merging operations \textbf{ADD}, \textbf{AVG} and \textbf{MAX} are available for both ALO and SLO methods.
As a baseline, we use the VGG16 network without any side output but only the final output, called No Side Outputs~(\textbf{NSO}).



\subsection{Training results - Methods Comparison}

The first test set was design to identify the best neural network and its best merging methods. We train all nets with all merging methods for 100 epochs to determine which one learns faster and achieves the best results. This conduct lead us to understand how layers can be easily combined to produce outputs with good precision.

Figures \ref{fig:validation_accuracy_pixel_error} presents the relevant curves obtained during the training phase for the proposed approaches. Figure \ref{fig:validation_accuracy_pixel_error}, presents fuse pixel error loss for tested approaches. ALO networks appear to be more stable with a faster decay than NSO and all SLO approaches. Also, it is important to notice that NSO and SLO-MAX produce high instability in the learning course (SLO-MAX seems to overfit around 40 epochs). On the other hand, ALO-AVG presents the best result for the test, followed by ALO-MAX and ALO-ADD merging strategies.

\begin{figure}
  \caption{Pixel Error Validation Loss \remEwai{Only Pixel Error according to the text}}
  \centering
  %\begin{tabular}{ll}
    \includegraphics[width=1.\columnwidth]{figures/falreis/pixel_error.png}
  
    %\includegraphics[width=1.\columnwidth]{figures/falreis/validation_accuracy.png}
  %\end{tabular}
  \label{fig:validation_accuracy_pixel_error}
\end{figure}

From previous graph, it is possible to conclude that ALO networks had superior and more desirable behavior than the SLO and NSO models. It is believed that these results are consequence of the considerably larger amount of side outputs, which create more possibilities of interchangeability between confident values.

\subsection{Side outputs contribution in each merging strategy}
\label{ssec:merging_learn}

For each merging strategy, each layer learns different information. The merging method influences how the network learns. It is possible to see how each side-output contributes to the final output in Figure \ref{fig:side_outputs}. To simplify the study of side outputs, we decided to visualize only the last  output from each stage in \textbf{ALO} network (that contains 13 side outputs). \remEwai{Not clear relatively to Fig. 2} To make the results more clear, images were also converted to black and white, where white pixels were classified as road and black pixels were classified as background.

\begin{figure*}
  \caption{Side outputs for each merging strategy in ALO network.}
  \centering
  \includegraphics[width=0.9\textwidth]{figures/falreis/side_outputs.png}
  \label{fig:side_outputs}
\end{figure*}

\remEwai{Fig. \ref{fig:side_outputs} is not so clear. Notation from the text should be reused: the side output i what is denoted $\mathcal{H}_i$ (if I'm correctly understanding)}

Figure \ref{fig:side_outputs} indicates that the first two side outputs does not produce significant information. Images are almost white, indicating that all pixels were classified as road. ALO-AVG and ALO-ADD third layer contains a clear separation between road pixel than non-road pixels. ALO-MAX's third layer, on the other hand, does not clearly separate road from non-road pixels. The results are almost pixelated when compared with the original image.

Figure \ref{fig:side_outputs} also indicates that fourth layer clearly contains the best side output for the evaluated networks. The road marks are clearly visible, but with some noise. ALO-MAX contains a lot of noise, while ALO-ADD contains a few ones. The final side output contains a lot of noise, with results far away worse than the previous layer. This possibly indicates that the layer was not able to correctly learn the information from the previous one.

The fuse layer in ALO network with different strategies take all information from side outputs and combine them to produce a single output \remEwai{= the proposition $Z$}. Since the network was trained before \remEwai{? Not clear}, merging layers \remEwai{merging layers = fuse layer?} seems to accept bad results and use them to produce good values.

\subsection{Best results}

In order to improve the results, a new set of tests were performed using 500 training epochs. As some networks had a poor performance in the previous test and other tests with different parameters, we decided to evaluate only ALO network in this new round of tests. The categorical cross entropy validation accuracy and pixel error validation loss for ALO nets are available in Figure \ref{fig:val_acc_500_epochs}.

\begin{figure*}
  \caption{Categorical Cross Entropy Validation Accuracy and Pixel Error Validation Loss results for 500 epochs test set}
  \centering
  \begin{tabular}{ll}
    \includegraphics[width=1.\columnwidth]{figures/falreis/val_acc_500_epochs.png}
  
    \includegraphics[width=1.\columnwidth]{figures/falreis/pixel_error_500_epochs.png}
  \end{tabular}%
  \label{fig:val_acc_500_epochs}
\end{figure*}

The best results of both metrics are quite similar for all networks. This indicates absence of a far better method to combine side outputs. The best result for cross entropy validation metric is just \textbf{0.0009} above the worst one (0.0332 for ALO-AVG and 0.0372 for ALO-MAX). For pixel-error validation loss, the best value is just \textbf{0.0040} above the worst one (0.983 for ALO-ADD and 0.9821 for ALO-AVG).

Due to the similarity of the results, we will indicate the best method using the value of validation pixel-error loss metric. For this criteria, ALO-AVG was defined as the best method of our training set.

\subsection{Post-processing using mathematical morphology}

\remEwai{Insert reference to section IV that describes that post-processing. \cite{najman13} should go in section IV.}
After the training procedure, we create a post processing step to reduce possible noises in results proposition. For this, we used the mathematical morphology operation of Opening~\cite{najman13}. This procedure removes small noises created by the foreground~(the road) in the background. We defined a set of kernels with the sizes of $5\times5$, $7\times7$, $9\times9$, $11\times11$ and $13\times13$ and applied them in the images to reduce different sizes of noises. 
\remEwai{I don't understand here. All the structuring elements are applied sequentially? which makes no sense since the opening by the largest structuring element is included in the opening by the smallest structuring element if I'm right. Or are they all tested independently? In this latter case, which size is finally retained?}

A simple comparison is presented in Figure \ref{fig:post_processing_comp}. In this image, we selected an output result that clearly shows the benefits of mathematical morphology post processing. It is possible to see the removal of part of the noise in the far right of the image (\textit{white pixels}). This procedure increases the confidence, as small variations in the results can lead to a potential problem, if used in a self-driving vehicle. 

A side effect of this method is the removal of some points that seems to fit correctly. This situation happens frequently in the base of the road proposition. In Figure \ref{fig:post_processing_comp}, it can be seen in the bottom left and the bottom right of the road (\textit{red pixels}).

\begin{figure}
  \caption{Comparison between ALO-AVG without post processing and ALO-AVG with post-processing with mathematical morphology. In the last picture, \textit{white} pixels represents desirable differences while \textit{red} pixels represents undesirable ones.}
  \centering
  \includegraphics[width=1.\columnwidth]{figures/falreis/post_processing_comparison.png}
  \label{fig:post_processing_comp}
\end{figure}

\subsection{Evaluation results and comparison with the state-of-the-art}

Reminding that the test evaluation could only be performed using KITTI Server, the metrics provided are maximum F1-measure~(MaxF), average precision~(AP), precision~(PRE), recall~(REC), false positive rate~(FPR) and false negative rate~(FNR). 

The server tests were performed using ALO-AVG method, the best one in the training process. To provide succint labels, we will use the name \textbf{ALO-AVG} for the regular approach and \textbf{ALO-AVG-MM} for the version with mathematical morphology post-processing. 

The results achieved  on the test set according to each category in the road scenes are presented in Table~\ref{tab:metrics}. As expected, the ALO-AVG-MM model performs better then the ALO-AVG in almost all the cases. {\color{red}It is also possible to notice that although the post-processing slightly improves the overall performance, it also increases the number of false negatives. This could be an indication that perhaps the applied kernel sizes are not adequate and are removing more of the foreground than desired.}

If compared with the state-of-the-art~(called \textit{PLARD}, an anonymous submission on the KITTI Server platform), the proposed method is comparable and sometimes superior, regarding the maximum F1-measure and the recall metrics. This is due to the fact that although the reported state-of-the-art on the dataset presents a superior average precision, it also almost always presents a higher rate of false positives and negatives. This indicates that the proposed methods are more precise in delineating the regions to be segmented.

\begin{table}
  \scriptsize
  \caption{KITTI benchmark evaluation results for ALO-AVG-MM}
  \renewcommand{\arraystretch}{1.2}
  \begin{tabular}{{l}{c}{c}{c}{c}{c}{c}}
    \hline
    Benchmark & MaxF & AP & PRE & REC & FPR & FNR 
    \\
    \hline
    UM\_ROAD & 91.15\% & 83.82\% & 89.07\% & 93.33\% & 5.22\% & 6.67\%
    \\
    UMM\_ROAD & 94.05\% & 90.96\% & 94.82\% & 93.29\% & 5.60\% & 6.71\%
    \\
    UU\_ROAD & 89.45\% & 79.87\% & 85.40\% & 93.90\% & 5.23\% & 6.10\%
    \\
    URBAN\_ROAD & 92.03\% & 85.64\% & 90.65\% & 93.45\% & 5.31\% & 6.55\%
    \\
    \hline
  \end{tabular}
  \label{tab:metrics}
\end{table}

A visual representation of the results is presented in Figure \ref{fig:visual_representation}. This image shows the results marked (in green) over the road, to show the performance of our model.

\begin{figure}
  \caption{Visual representation of the results}
  \centering
  \includegraphics[width=1.\columnwidth]{figures/falreis/visual_representation.png}
  \label{fig:visual_representation}
\end{figure}

\section{Conclusion}
\label{sec:conclusion}

This work addressed the problem of merging side outputs extracted from the convolutional layer model VGG to create region propositions for the task of image segmentation. It was proposed to use a $max()$ function to enhance confident values during training to be evaluated using a cross-entropy loss function. It was also studied the impact that the number of side outputs have on the proposed strategy and if a simple mathematical morphology operation could enhance the performance on the task. 

Experiments demonstrated that the $max()$ function is viable for merging maps with different sizes and connotations, and could place the proposed strategy among the state-of-the-art approaches for the task on the Kitti dataset. It was also demonstrated that a large amount of side outputs increases the network confusion during the training step, but could also create jumps that could lead to better performance, in terms of accuracy. The post-processing strategy slightly improved the performance, but requires further studies.

This research opens novel opportunities for study such as: (i) exploring different merging functions, less susceptible a values fluctuations;  (ii) explore regularization techniques to sustain larger amounts of side outputs consistent; and (iii) insert the mathematical morphology kernels on the learning process to search for the best kernel size. 
 
The code  and a file containing all dependencies to reproduce the experiments is public available online in \url{https://github.com/falreis/segmentation-eval}. 


\bibliographystyle{IEEEtran}
\bibliography{2019-ijcnn,dl,hier-seg}

\end{document}
