\section{Introduction}
\label{sec:intro}

Image segmentation refers to the partition of an image into a set of regions representing  meaningful areas. It is considered a challenging semantic task aiming to determine and group uniform regions for analysis. {\color{green}reescrito Felipe - mantido mesma estrutura DOMINGUEZ }According to~\cite{DOMINGUEZ}, to create an adequate segmented image it is necessary that the output presents some fundamental characteristics, such as: (i) region uniformity and homogeneity in  its features, aka. gray level, color, or texture; (ii) region continuity, without holes; (iii) significant difference to adjacency regions; and (iv) spacially accurationess with smoothness, without raggedness. Segmentation is an active topic of research and in a traditional approach the task is performed using hand-engineered features \cite{Segnet:2017:7803544}.  

\textbf{Insert scale-set theory reference to characterize hand-eng. Talvez retirar o paragrafo do deep para introduzir o related}

Recently, deep learning architectures drastically changed the computational paradigm for visual tasks. The main advantage of deep learning algorithms is that it does not require an engineered model to operate, meaning that they are capable of learning not only the features to represent the data but also the models to describe it~\cite{goodfellow16}. The success of these approaches depends on a set of standards learned by the network, in which more complex concepts are built from simpler ones. In the deep learning approach applied in images, the raw pixel on the input layer is learned as segments and parts until the composition of multiple object concepts later in the network.

Unsurprisingly, many approaches have been proposed in the recent years, as \cite{Xie:2017:HED:3158436.3158453} \cite{RCF:8100105} \cite{COB:7917294}, to explore information from the outputs of different layers of a deep learning network, producing different maps. One challenge in this strategy is how to combine these maps, to produce an unique output that could represent different concepts and fits the goal of the network. In this work it is presented a simple strategy to combine maps from the layer outputs and create region proposition for the task of binary image segmentation.   

The remainder of this work is organized as it follows: Section \ref{sec:related} contains the related works; Section \ref{sec:method} describes the method developed to produce segmentation; Section \ref{sec:experiments} shows the dataset description, the experimental setup and results the experiments; and, finally, Section \ref{sec:conclusion} concludes this paper.

