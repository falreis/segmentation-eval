\section{Experiments}
\label{sec:experiments}

bla bla bla

\subsection{The Kitti dataset}

KITTI Vision Benchmarking Suite is a project of Karlsruhe Institute of Technology and Toyota Technological Institute at Chicago to provide a real-world computer vision benchmark for autonomous driving platform Annieway. KITTI contains benchmarks and datasets for the following area of interests: stereo, optical flow, visual odometry, 3D object detection and 3D tracking.

One of the benchmarks in the KITTI suite is the Road/Lane Dataset Evaluation \cite{KITTI}. The road and lane estimation benchmark consists of 289 training and 290 test images, in four different categories of road scenes \cite{KITTI}:

\begin{itemize}
 \item uu - urban unmarked (98 training images and 100 test images) \cite{KITTI};
 \item um - urban marked (95 training images and 96 test images) \cite{KITTI};
 \item umm - urban multiple marked lanes (96 training images and 94 test images) \cite{KITTI};
 \item urban - combination of the three above \cite{KITTI}.
\end{itemize}

Ground truth has been generated by manual annotation of the images and is available for two different road terrain types: road - the road area (the composition of all lanes), and lane (the ego-lane, the lane the vehicle is currently driving on) \cite{KITTI} . Ground truth is provided for training images only \cite{KITTI}. 

As the dataset does not provide test ground truth, the results must be evaluated using a benchmarking tool provided with the dataset. This tool  performs road and lane estimation in the birds-eye-view space \cite{KITTI} . The metrics used are Maximum F1-measure, Average precision as used in PASCAL VOC challenges, Precision, Recall, False Positive Rate, False Negative Rate, F1 score and Hit Rate \cite{KITTI}.

\subsection{Experimental setup}

\subsection{Results}

\begin{figure*}[htb]
\begin{center}
\begin{tabular}{ll}
(2.a) Training accuracy rate&(2.b) Cross-entropy training loss\\
\resizebox{.49\textwidth}{!}{
	\begin{tikzpicture} 
		\begin{axis}[
			xlabel= Epochs,
			ylabel= Accuracy rate,
			xmin = 0,
			xmax = 400,
			ymin = 0,
			ymax = 1,
			width= 100mm,
			height = 70mm,
			%legend columns=-1,
			axis x line = bottom,
			axis y line = left,
			tickwidth = 0pt,
			axis line style = { black!30 },
			legend style = {draw = none, at={(.35,1)},anchor=north, cells={anchor=west}}
			]
			\addplot+[mark=none, blau_1b, very thick,smooth] file {figures/data/aug-add-train-acc.dat};
			\addplot+[mark=none, blau_2b, very thick,smooth] file {figures/data/aug-avg-train-acc.dat};
			\addplot+[mark=none, rot_8b, very thick,smooth] file {figures/data/aug-maj2-train-acc.dat};
			\addplot+[mark=none, rot_9b, very thick,smooth] file {figures/data/aug-maj3-train-acc.dat};
			\addplot+[mark=none, lila_10b, very thick,smooth] file {figures/data/aug-max-train-acc.dat};
			%\legend{add,avg,maj2,maj3,max}
		\end{axis} 
	\end{tikzpicture}
	
	}&
\input{figures/aug-line-loss-train}\\
(2.a) Overfitting check&(2.b) Cross-entropy validation loss\\
\input{figures/aug-line-acc-val}&
\input{figures/aug-line-loss-val}\\
\end{tabular}%
\caption{Aug - Learning curves for the compared approaches. Left panel displays the accuracy obtained on the training and validations sets. Right panel displays the cross-entropy objective function}
\end{center}
\label{fig:learning}
\end{figure*}

\begin{figure*}[htb]
\begin{center}
\begin{tabular}{ll}
(2.a) Training accuracy rate&(2.b) Cross-entropy training loss\\
\resizebox{.49\textwidth}{!}{
	\begin{tikzpicture} 
		\begin{axis}[
			xlabel= Epochs,
			ylabel= Accuracy rate,
			xmin = 0,
			xmax = 400,
			ymin = 0,
			ymax = 1,
			width= 100mm,
			height = 70mm,
			%legend columns=-1,
			axis x line = bottom,
			axis y line = left,
			tickwidth = 0pt,
			axis line style = { black!30 },
			legend style = {draw = none, at={(.35,1)},anchor=north, cells={anchor=west}}
			]
			\addplot+[mark=none, blau_1b, very thick,smooth] file {figures/data/noaug-add-train-acc.dat};
			\addplot+[mark=none, blau_2b, very thick,smooth] file {figures/data/noaug-avg-train-acc.dat};
			\addplot+[mark=none, rot_8b, very thick,smooth] file {figures/data/noaug-maj2-train-acc.dat};
			\addplot+[mark=none, rot_9b, very thick,smooth] file {figures/data/noaug-maj3-train-acc.dat};
			\addplot+[mark=none, lila_10b, very thick,smooth] file {figures/data/noaug-max-train-acc.dat};
			%\legend{add,avg,maj2,maj3,max}
		\end{axis} 
	\end{tikzpicture}
	
	}&
\input{figures/noaug-line-loss-train}\\
(2.a) Overfitting check&(2.b) Cross-entropy validation loss\\
\input{figures/noaug-line-acc-val}&
\resizebox{.49\textwidth}{!}{
	\begin{tikzpicture} 
		\begin{axis}[
			xlabel= Epochs,
			ylabel= Loss,
			xmin = 0,
			xmax = 400,
			ymin = 0,
			ymax = 2,
			width= 100mm,
			height = 70mm,
			%legend columns=-1,
			axis x line = bottom,
			axis y line = left,
			tickwidth = 0pt,
			axis line style = { black!30 },
			legend style = {draw = none, cells={anchor=west}}
			]
			\addplot+[mark=none, blau_1b, very thick,smooth] file {figures/data/noaug-add-val-loss.dat};
			\addplot+[mark=none, blau_2b, very thick,smooth] file {figures/data/noaug-avg-val-loss.dat};
			\addplot+[mark=none, rot_8b, very thick,smooth] file {figures/data/noaug-maj2-val-loss.dat};
			\addplot+[mark=none, rot_9b, very thick,smooth] file {figures/data/noaug-maj3-val-loss.dat};
			\addplot+[mark=none, lila_10b, very thick,smooth] file {figures/data/noaug-max-val-loss.dat};
			\addplot+[mark=none, green, very thick,smooth] file {figures/data/lr-v-acc.dat};
			\legend{add,avg,maj2,maj3,max}
		\end{axis} 
	\end{tikzpicture}

}\\
\end{tabular}%
\caption{No Aug - Learning curves for the compared approaches. Left panel displays the accuracy obtained on the training and validations sets. Right panel displays the cross-entropy objective function}
\end{center}
\label{fig:learning}
\end{figure*}

\subsection{Qualitative analysis}