\documentclass[10pt,twocolumn,letterpaper]{article}

\usepackage{cvpr}
\usepackage{times}
\usepackage{epsfig}
\usepackage{graphicx}
\usepackage{amsmath}
\usepackage{amssymb}

\def\cvprPaperID{1} % *** Enter the CVPR Paper ID here

\usepackage[breaklinks=true,bookmarks=false]{hyperref}

\cvprfinalcopy % Comment this line and it stop working! :(
\ifcvprfinal\pagestyle{empty}\fi

\def\httilde{\mbox{\tt\raisebox{-.5ex}{\symbol{126}}}}

% Pages are numbered in submission mode, and unnumbered in camera-ready
%\ifcvprfinal\pagestyle{empty}\fi
\setcounter{page}{1}

\graphicspath{ {./images/} } 

\sloppy

%-------------------------------------------------------------------------
%-------------------------------------------------------------------------

\begin{document}

%%%%%%%%% TITLE
\title{Convolutional Neural Network to Image Segmentation}

\author{Felipe Augusto Lima Reis\\
PUC Minas - Pontif\'icia Universidade Cat\'olica de Minas Gerais\\
R. Walter Ianni 255 - Bloco L - Belo Horizonte, MG, Brasil\\
{\tt\small falreis@sga.pucminas.br}
}

\maketitle
%\thispagestyle{empty}

%%%%%%%%% ABSTRACT
\begin{abstract}
   The ABSTRACT is to be in fully-justified italicized text, at the top
   of the left-hand column, below the author and affiliation
   information. Use the word ``Abstract'' as the title, in 12-point
   Times, boldface type, centered relative to the column, initially
   capitalized. The abstract is to be in 10-point, single-spaced type.
   Leave two blank lines after the Abstract, then begin the main text.
   Look at previous CVPR abstracts to get a feel for style and length.
\end{abstract}

%%%%%%%%% BODY TEXT
\section{Introduction}

Image segmentation refers to the partition of an image into a set of regions to cover it, to represent meaningful areas \cite{DOMINGUEZ}. The goal is to simplify and/or change the representation of an image into something
that is more meaningful and easier to analyze \cite{AHMED_SARMA}.

Segmentation has two main objectives: the first one is to decompose the image into parts for further analysis and the second one is to perform a change of representation \cite{DOMINGUEZ}. Also, segmentation must follow some characteristics to identify regions, as it follows:

\begin{itemize}
 \item Regions of an image segmentation should be uniform and homogeneous with respect to some characteristic, such as gray level, color, or texture \cite{DOMINGUEZ};
 \item Region interiors should be simple and without many small holes \cite{DOMINGUEZ};
 \item Adjacent regions of a segmentation should have significantly different values with respect to the characteristic on which they are uniform \cite{DOMINGUEZ};
 \item Boundaries of each segment should be smooth, not ragged, and should be spatially accurate \cite{DOMINGUEZ}.
\end{itemize}

This paper evaluates segmentation methods using Deep Neural Networks and compares with classical methods of segmentation, using the superpixels approach. Also, the paper evaluates the composition of classical methods with DNN approach, to speed up the training process and become more accurate.

The organization of this paper is as follows. In the next Section we discuss some related work and segmentations methods.  In Section \ref{sec:mat_metodos} its explained the method used in this paper. Then in Section \ref{sec:testes} we present an the results of the tests made for this paper and analise them. In Section \label{sec:conclusao} conclude the work.

%-------------------------------------------------------------------------
\section{Related Work} \label{sec:ref_teorico}

Superpixels are the result of perceptual grouping of pixels, or seen the other way around, the results of an image oversegmentation \cite{WANG201728}. One example of oversegmentation can be seen in Figure \ref{fig:superpixel}. Superpixel can cause substantial speed-up of subsequent processing since the number of superpixels of an image reduces in contrast with the original number of pixels \cite{WANG201728}.
 
\begin{figure}[ht]
  \centering
  \includegraphics[width=0.48\textwidth]{superpixels.png}
  \caption{Segmented images using SLICO and EGB superpixels}
  \label{fig:superpixel}
\end{figure}

%-------------------------------------------------------------------------
\section{Segmentation Method} \label{sec:mat_metodos}

%-------------------------------------------------------------------------
\section{Tests and Results} \label{sec:testes}

%-------------------------------------------------------------------------
\section{Conclusion}

%-------------------------------------------------------------------------

{\small
\bibliographystyle{ieee}
\bibliography{egbib}
}

\end{document}
